% \cvsection{研究经历}

% \begin{cventries}
% % \vspace{-1.0mm}

% \cvexperience
% {\entrylocationstyle{研究助理},蔡淑琴教授,管理学院,华中科技大学}
% {09/2015 - PRESENT}
% {
%     \begin{cvitems}
%     \item {结合句法结构与向量空间模型,提出一种考虑抱怨问题路径的网络抱怨识别方法。}
%     \item {考虑知识、情感和互动三个资源维度,建立处理在线负面口碑的专家识别方法。}
%     \item {参与大数据实验室的建设,撰写17年国家自科基金项目申请书中关于大数据产品质量测度的部分。}
%     \item {基于大数据交易平台,建立大数据产品价值评价的质量模型、质量识别和价值评价方法。}
%     \end{cvitems}
% }

% \cvexperience
% {\entrylocationstyle{研究助理},大数据实验室,管理学院,华中科技大学}
% {03/2016 - 08/2017}
% {
%     \begin{cvitems}
%     \item {利用有源标签信号强度在多个接收器的差别,实现基于ZigBee的区域定位接口。}
%     \item {搭建基于Hadoop的全分布式计算机集群,实现Map/Reduce框架下的Naïve Bayes分类器。}
%     \end{cvitems}
% }

% \cvexperience
% {\entrylocationstyle{研究助理},栾丽霞教授,公共管理学院 \& 体育部,华中科技大学}
% {04/2017 - 05/2017}
% {
%     \begin{cvitems}
%     \item {结合学生体质健康标准数据与高水平运动员体质特征,提出一种半监督学习下的体育选材推荐方法。}
%     \end{cvitems}
% }

% \cvexperience
% {\entrylocationstyle{研究实习},罗铁坚教授,计算机与控制学院,中国科学院大学}
% {06/2017 - 07/2017}
% {
%     \begin{cvitems}
%     \item {评阅与重现Facebook人工智能研究院提出的端到端谈判对话机器人。}
%     \item {面向智能客服问答领域,改进基于拷贝和检索的自然答案生成系统与TREC/QA评价体系。}
%     \end{cvitems}
% }

% \cvexperience
% {\entrylocationstyle{研究实习},何琨教授,John Hopcroft Lab \& 创新研究院,华中科技大学}
% {07/2017 - PRESENT}
% {
%     \begin{cvitems}
%     \item {针对企业经济事项的会计人工处理问题,建立一种端到端的账务智能处理框架。}
%     \item {提出会计事项的机器理解方法,利用Word2Vec方法实现会计事项的词向量空间嵌入。}
%     \item {提出会计分录的机器编制方法,利用GRUs+Attention机制实现会计知识的深度学习。}
%     \end{cvitems}
% }

% \end{cventries}