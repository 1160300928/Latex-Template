\usepackage[margin=1in]{geometry}
\pagestyle{headings}
% 调整页面大小,默认页面与常用规格不符

% 用来输入摘要
\usepackage{abstract}

% 用来插入图片
\usepackage{graphicx}

% 插入数学符号
\usepackage{amsmath}
\usepackage{amsfonts}
\usepackage{amssymb}
\usepackage{amsthm}

% 引用链接可点击
\usepackage{cite}
\usepackage[colorlinks,linkcolor=black,anchorcolor=blue,citecolor=green]{hyperref}

% 用于插入伪代码
\usepackage{clrscode3e}

% 用于插入C代码
\usepackage{listings}

% 调整代码的背景和风格
\usepackage{xcolor}
\lstset{
    %行号
    numbers=left,
    %背景框
    framexleftmargin=10mm,
    frame=none,
    %背景色
    %backgroundcolor=\color[rgb]{1,1,0.76},
    backgroundcolor=\color[RGB]{245,245,244},
    %样式
    keywordstyle=\bf\color{blue},
    identifierstyle=\bf,
    numberstyle=\color[RGB]{0,192,192},
    commentstyle=\it\color[RGB]{0,96,96},
    stringstyle=\rmfamily\slshape\color[RGB]{128,0,0},
    %显示空格
    showstringspaces=false
}

% 设置参考文献风格
\bibliographystyle{plain}

% 处理下划线
\usepackage{underscore}

\usepackage{ctexheading}

% 设置Section标题风格
% 标题居左《一、标题》
%\ctexset{section={name={,、},number= \chinese{section},format = \Large\bfseries\raggedright}}
% 标题居左《第一章 标题》
\ctexset{section={name={第,章},number= \chinese{section},format = \Large\bfseries\raggedright}}
% 标题居中《第一章 标题》
%\ctexset{section={name={第,章},number= \chinese{section}}} 