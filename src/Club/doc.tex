\documentclass{club}

\infoName{IBM俱乐部}
\begin{document}
\maketitle
\tableofcontents
\newpage

\section{总则}

\subsection{社团宗旨}

哈尔滨工业大学IBM俱乐部,本着服务于计算机系教学和同学们发展的目标,在哈尔滨工业大学人才培养的理念的指导下,通过发动和开展各种贴近计算机专业的活动,让同学们对计算机领域知识有更全面、更透彻的认识,并通过学会活动突出并强化专业的理论知识、弥补教学中实践部分,获得理论联系实际的实践机会,培养出具备综合能力和素质,适应社会主义市场经济发展和现代化建设需要,培养专业素养与专业能力结合、知识学习与实践能力并重、诚信做人与创新能力兼备的、立足专业、培养服务于现代社会发展的复合型高级应用人才。

\subsubsection{活动理念}

IBM俱乐部,作为计算机学科靠实践、训练或反复体验而获得才学的人组成的学术团体。
意在:扩展知识、锻炼能力、启发责任心、激发创造力。学即“课堂不学”,会即“职业应会”。
注重思维培养 ,边学、边练、边思考、边修正、边提高。乐在其中,有所收益。

在老师的指引下,以学生的自主学习为主,多鼓励和提倡扩展学习。
俱乐部的学术研究侧重重大项目的跟踪报告,前沿知识的搜集和整理。
并定期形成一定的学习报告,谈论问题,形成最终的研究报告。

社会实践,则是学以致用,实施的的过程。寻找真实的项目进行锻炼,溶知识与现实一体。

\subsubsection{培养理念}

\paragraph{坚持专业素养于专业能力结合}俱乐部时代发展,发挥计算机系的优势,参与科研项目与商业项目,把专业知识内化为基本能力与素质。
努力培养基础扎实,知识广博,有获取新知识能力,具有较强竞争意识和合作精神的应用人才。

\paragraph{坚持知识学习与实践能力并重}大学生实践能力的培养日益受到人们的重视,因为实践是创新的基础。应该彻底改变传统教育模式下实践教学处于从属地位的状况。
构建科学合理培养方案的一个重要任务是必须为学生构筑一个合理的实践能力体系,并从整体上策划每个实践教学环节。
这种实践教学体系是与理论教学平行而又相互协调、相辅相成的。
俱乐部活动的过程中安排了专业策划人士的讲座和教授的指导,尽可能为学生提供综合性、设计性、创造性比较强的实践环境,如果每个大学生在4年中能经过多个这种实践环节的培养和训练,不仅能培养学生扎实的基本技能与实践能力,而且对提高学生的综合素质大有好处。

\paragraph{坚持诚信做人与创新能力兼备}诚信求实,旨在对待学术和活动的精益求精和认真对待。
严守哈工大“规格严格,功夫到家”的校训,以踏实的态度对待一切的学术和工程工作,把握时代脉搏,提高创新能力,迎接时代的挑战。

\subsection{社团目标}

本俱乐部立足于实验室,依托于IBM公司,为广大同学们特别是计算机系的同学们,提供合作、交流、学习、实践的大平台,以此为基础,提高同学们的科研水平、实践能力、学习能力。

\section{组织机构设置及职能}

IBM俱乐部设有全体会员大会和部长级会议。

\paragraph{第一条}IBM俱乐部全体会员大会是该组织的最高权利机构。
\paragraph{第二条}IBM俱乐部全体会员大会行使组织内部规章制度的制定权。
\paragraph{第三条}IBM俱乐部全体会员大会每月召开一次例会,如果有必要,或有半数以上会员提议,可以临时召开全体会员大会。俱乐部全体会员大会召开的时候,由社长或社长指定的人员主持。
\paragraph{第四条}IBM俱乐部全体会员大会行使下列职权:
\begin{enumerate}
    \item 修改组织的规章制度;
    \item 监督组织规章制度的实施;
    \item 选举俱乐部的副部长;
    \item 根据部长的提名,决定俱乐部内其他职务人员;
    \item 监督组织内的经费使用情况;
    \item 决定俱乐部内各个职务的设置;
\end{enumerate}
\paragraph{第五条}IBM俱乐部全体会员大会有权罢免特别活动筹备负责人。
\paragraph{第六条}IBM俱乐部章程的修改,由指导老师建议、或由俱乐部社长提议,并由俱乐部全体会员大会以全体会员的半数或大多数通过。
\paragraph{第七条}每周举行一次部长级会议。
\paragraph{第八条}部长的任免由部长级会议全体人员的半数或大多数人员的同意。
\paragraph{第九条}IBM俱乐部部长行使下列职权:
\begin{enumerate}
    \item 解释组织的章程、监督章程的实施;
    \item 制定和修改应当章程以外的规章制度;
    \item 解释组织内的规章制度;
    \item 监督会员的工作;
    \item 决定同外界组织或个人缔结的协议或重要文件的批准和废除;
    \item 规定和决定向会员授予组织内的荣誉称号,发给奖励,通报表扬或向学校推荐会员获得荣誉。
\end{enumerate}
\paragraph{第十条}IBM俱乐部部长职能是:
\begin{enumerate}
    \item 科学分配工作任务;
    \item 协调会员之间的工作关系;
    \item 量化测评队员的思想状况,工作及考勤情况;
    \item 监督全体会员遵守组织的章程和各项规章制度;
    \item 管理组织内部的档案和文档资料;
    \item 拟定工作计划,社会实践活动的计划;
    \item 活动结束后进行总结,并整理出文字资料;
    \item 组织会员写活动报告;
    \item 评审并录用优秀的部员或部长;
    \item 负责社团与外界组织或个人的交流活动。
\end{enumerate}
\paragraph{第十一条}各部门职能
\subparagraph{外联部}
\begin{enumerate}
    \item 为活动拉取赞助
    \item 加强与外校的技术类社团的联系与交流
    \item 负责与本校相关组织(IBM实验室、关毅实验室等)的交流与合作
\end{enumerate}
\subparagraph{宣传部}
\begin{enumerate}
    \item 全面负责本社团的海报、展版及网页等各种宣传工作
    \item 配合各部门做好各项活动的宣传
    \item 不定期的面向全校及社会各界宣传本社团,使社团的各项活动更加透明化
\end{enumerate}
\subparagraph{组织部}
\begin{enumerate}
    \item 负责管理会员并处理会员违纪事件,并收集会员对本社团的建议和批评
    \item 负责对各部门的人员考查,并向主席团提出书面报告
    \item 负责本社团活动的构思和策划,并按照学校有关规定在活动前提交书面计划,在活动得到批准后全面组织使活动顺利进行,以达到预定目的
\end{enumerate}
\subparagraph{财务部}
\begin{enumerate}
    \item 负责本社团的财务工作,负责保管本社团的财产、财务登记和检查
    \item 购置必需品,明确资金收支情况
\end{enumerate}
\subparagraph{技术部}
\begin{enumerate}
    \item 全面负责俱乐部技术类活动
    \item 协助宣传部制作宣传海报、视频
    \item 有偿接取学校及社会各类外包活动
    \item 负责俱乐部成员的学术研究活动、包括项目评估、相关知识的搜集和整理等。
\end{enumerate}

\section{组织管理制度}

\paragraph{第一条} 为使本社更加稳定有序的运作和发展,特制定如下管理条例

\paragraph{第二条} 会员管理制度
\begin{enumerate}
    \item 部员必须积极参加与本协会和活动,不迟到、不早退,有事要事先请假。
    \item 部员在参加有关课程、会议及活动方面,若迟到、早退两次计一次缺席,累计无故两次缺席予以会内警告,四次予以留会察看处分,超过四次给予勒令退会处分。
    \item 每次会议及活动前,部员进行签到制度,考勤人员进行点名记录,并将考勤情况记入部员考勤记录表中。
    \item 凡未和协会协商擅自退会的部员视为自动退会处理,并给予开除协会处分。
    \item 所有部员有义务高质高效完成俱乐部所交任务。
    \item 每学期末对优秀社团干部,优秀部员\footnote{由部员参加活动的积极性、工作完成情况、工作态度等进行打分,根据分数优先享有各种福利。}给予一定的物质奖励。奖励方法由部长决定签字后统一划拨。
\end{enumerate}

\paragraph{第三条} 会议制度
\begin{enumerate}
    \item 每月召开一次例会。
    \item 每次会议及活动前,部员进行签到制度,考勤人员进行点名记录,并将考勤情况记入会员考勤记录表中。
\end{enumerate}

\paragraph{第四条} 活动制度
\begin{enumerate}
    \item 部员应积极参加俱乐部各项活动。
    \item 活动形式及内容由部员商议决定。
    \item 每年进行一次实验室分流,给不同兴趣的部员选择实验室的机会。
    \item 每年举行一次社团内部的大型联欢晚会或外游活动,增进会员感情。
\end{enumerate}

\paragraph{第五条} 纳新制度
\begin{enumerate}
    \item 为促进本俱乐部长远发展,不断开创各项工作的新局面,本社每年都将在全校范围内选拔吸收新的会员加入组织。纳新对象主要是全日制的低年级学生,兼收少量优秀在校高年级学生,纳新的时间、方式、人数、期限由本会会员大会决定。俱乐部的纳新工作原则上在每年的上学年开学初进行,如有需要,可临时决定招收新成员,有关情况必须在俱乐部内妥为公布。
    \item 本社纳新工作坚持按照公开公正择优录取的原则,任何纳新工作人员徇私舞弊的行为,都必须予以追究。
    \item 社团纳新工作严格按照通告报名、接受申请、组织面试、试用考察、正式录用的程序进行。
    \item 申请加入俱乐部的学生\footnote{
        \begin{enumerate}
            \item 宣传部有一定的PS、摄影基础的优先录取,或有较强学习意愿的,也可酌情录取
            \item 外联部要求五官端正、口齿伶俐、表达能力强的优先录取
            \item 组织部、财务部成员由部长指定或全体会员大会决定
            \item 招新过程中,有某方面特别突出的,也可以破格录取
        \end{enumerate}}必须满足以下基本条件:热爱祖国,思想品质好,政治极上进,学习认真刻苦,工作勤勉踏实,有团队精神,为整个组织出谋划策.申请加入的同学经录用后,建立会员档案。
\end{enumerate}

\paragraph{第六条} 退出制度                                      
\begin{enumerate}
    \item 部员在本社工作满一年可申请退出;未满一年,原则上不可退出。部员长期不遵守组织规章制度或不能胜任在组织内的工作或已经不具备做一名本俱乐部成员的基本条件的时候,可自动申请退出或强制其退出。
    \item 部员申请退出的,必须递交申请书,经部长级会议批准后正式退出。
\end{enumerate}



\section{经费来源}

\paragraph{第一条} 本社经费来源于学校拨款,企事业单位赞助及外包活动收入。
\paragraph{第二条} 拉的赞助的资金存档管理,通报部长,并和会费收入统一由财务部存入专用帐户。
\paragraph{第三条} 每次活动经费由各部报告申请,经部长签字后划拨一定经费,并将相关票据凭证交于财务部,部长核对登记。
\paragraph{第四条} 财务情况公开透明,财务部每月向部长汇报财务状况,并接受所有社员监督。

\section{社团活动}
% 大致活动安排,包含预计活动时间、活动内容设想、活动规模等

\subsection{招新}

\subsection{俱乐部网站维护}

\subsection{WI输入法维护}

\subsection{实验室选拔}

\subsection{学期末总结}

% \appendix

% \section{社团花名册}

\end{document}